% Latex template: mahmoud.s.fahmy@students.kasralainy.edu.eg
% For more details: https://www.sharelatex.com/learn/Beamer

\documentclass{beamer}					% Document class

\usepackage[english]{babel}				% Set language
\usepackage[utf8x]{inputenc}			% Set encoding

\mode<presentation>						% Set options
{
  \usetheme{default}					% Set theme
  \usecolortheme{default} 				% Set colors
  \usefonttheme{default}  				% Set font theme
  \setbeamertemplate{caption}[numbered]	% Set caption to be numbered
}

% Uncomment this to have the outline at the beginning of each section highlighted.
%\AtBeginSection[]
%{
%  \begin{frame}{Outline}
%    \tableofcontents[currentsection]
%  \end{frame}
%}

\usepackage{graphicx}					% For including figures
\usepackage{booktabs}					% For table rules
\usepackage{hyperref}					% For cross-referencing

\title{Optimizing Saving Decisions in Counter-Strike 2}	% Presentation title
\author{Lou Zhou}								% Presentation author
\institute{Rice University}					% Author affiliation
\date{\today}									% Today's date	

\begin{document}

% Title page
% This page includes the informations defined earlier including title, author/s, affiliation/s and the date
\begin{frame}
  \titlepage
\end{frame}


% The following is the most frequently used slide types in beamer
% The slide structure is as follows:
%
%\begin{frame}{<slide-title>}
%	<content>
%\end{frame}

\section{Section One}

\begin{frame}{Counter-Strike 2}
	5-on-5, round-based first-person shooter
    \begin{itemize}
		\item First to 13 rounds wins the match
		\item Terrorist Side (Offensive) looks to plant and defend a bomb
    \item Counter-Terrorist (Defensive) seek to defend the site / defuse the bomb
		\item If you survive a round, weapons / utilities are carried over to the next round
	\end{itemize}
  Ability to use powerful weapons depends on the economy
  \begin{itemize}
    \item Money depends on what happens during previous rounds (e.g. winning or losing the round, getting kills)
    \item Being able to properly manage your economy is important to winning the game
  \end{itemize}
\end{frame}

\begin{frame}{Data}
	Matches are saved by the game via "demo" files
  \begin{itemize}
    \item Demo files can be parsed via the awpy package to produce full tracking and event data
    \item We use all demo files(259 matchups, 500+ matches) from tournaments during the later-half of the 2025 year
    \item Demo files are available to be downloaded on HLTV.org, web-scraping to automatically download
  \end{itemize}
\end{frame}

\begin{frame}{Saving}
  The idea of intentionally giving up a round to "save" powerful weapons for use in future rounds
	\begin{itemize}
    \item Useful for situations when the round is likely lost, but still holding on to these weapons
    \item Recently, in the professional scene, saving has been used substantially more liberally (e.g. in 5v3 situations)
  \end{itemize}
  We look to build a win-probability framework to determine whether this the decision is optimal
\end{frame}



\begin{frame}{Win Probability Models}
	Building Markov-Chain win-probability model for each round
  \begin{itemize}
    \item Given the score, economic situations of each team, what is the probability that team $x$ wins?
  \end{itemize}
  With these win-probabilities, we build an in-round MDP for saving
  \begin{itemize}
    \item The MDP will look to reach certain terminating state
    \begin{itemize}
      \item e.g. Terrorist Side won, Counter-Terrorists hold on to \$5000 worth of inventory
      \item These terminating states will then be used to calculate the state for our Markov-Chain to derive a win-probability
    \end{itemize}
    \item State - number of players left, value of current inventories, whether bomb has been planted, time left
    \item Action - whether to save or continue to go for it
    \item Value - the win-probabilities from potential ending Markov Chain states
  \end{itemize}
  From here, we can determine if / when it would be optimal to save
\end{frame}



\end{document}
